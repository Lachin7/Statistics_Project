% Options for packages loaded elsewhere
\PassOptionsToPackage{unicode}{hyperref}
\PassOptionsToPackage{hyphens}{url}
%
\documentclass[
]{article}
\title{statistics}
\author{}
\date{\vspace{-2.5em}}

\usepackage{amsmath,amssymb}
\usepackage{lmodern}
\usepackage{iftex}
\ifPDFTeX
  \usepackage[T1]{fontenc}
  \usepackage[utf8]{inputenc}
  \usepackage{textcomp} % provide euro and other symbols
\else % if luatex or xetex
  \usepackage{unicode-math}
  \defaultfontfeatures{Scale=MatchLowercase}
  \defaultfontfeatures[\rmfamily]{Ligatures=TeX,Scale=1}
\fi
% Use upquote if available, for straight quotes in verbatim environments
\IfFileExists{upquote.sty}{\usepackage{upquote}}{}
\IfFileExists{microtype.sty}{% use microtype if available
  \usepackage[]{microtype}
  \UseMicrotypeSet[protrusion]{basicmath} % disable protrusion for tt fonts
}{}
\makeatletter
\@ifundefined{KOMAClassName}{% if non-KOMA class
  \IfFileExists{parskip.sty}{%
    \usepackage{parskip}
  }{% else
    \setlength{\parindent}{0pt}
    \setlength{\parskip}{6pt plus 2pt minus 1pt}}
}{% if KOMA class
  \KOMAoptions{parskip=half}}
\makeatother
\usepackage{xcolor}
\IfFileExists{xurl.sty}{\usepackage{xurl}}{} % add URL line breaks if available
\IfFileExists{bookmark.sty}{\usepackage{bookmark}}{\usepackage{hyperref}}
\hypersetup{
  pdftitle={statistics},
  hidelinks,
  pdfcreator={LaTeX via pandoc}}
\urlstyle{same} % disable monospaced font for URLs
\usepackage[margin=1in]{geometry}
\usepackage{color}
\usepackage{fancyvrb}
\newcommand{\VerbBar}{|}
\newcommand{\VERB}{\Verb[commandchars=\\\{\}]}
\DefineVerbatimEnvironment{Highlighting}{Verbatim}{commandchars=\\\{\}}
% Add ',fontsize=\small' for more characters per line
\usepackage{framed}
\definecolor{shadecolor}{RGB}{248,248,248}
\newenvironment{Shaded}{\begin{snugshade}}{\end{snugshade}}
\newcommand{\AlertTok}[1]{\textcolor[rgb]{0.94,0.16,0.16}{#1}}
\newcommand{\AnnotationTok}[1]{\textcolor[rgb]{0.56,0.35,0.01}{\textbf{\textit{#1}}}}
\newcommand{\AttributeTok}[1]{\textcolor[rgb]{0.77,0.63,0.00}{#1}}
\newcommand{\BaseNTok}[1]{\textcolor[rgb]{0.00,0.00,0.81}{#1}}
\newcommand{\BuiltInTok}[1]{#1}
\newcommand{\CharTok}[1]{\textcolor[rgb]{0.31,0.60,0.02}{#1}}
\newcommand{\CommentTok}[1]{\textcolor[rgb]{0.56,0.35,0.01}{\textit{#1}}}
\newcommand{\CommentVarTok}[1]{\textcolor[rgb]{0.56,0.35,0.01}{\textbf{\textit{#1}}}}
\newcommand{\ConstantTok}[1]{\textcolor[rgb]{0.00,0.00,0.00}{#1}}
\newcommand{\ControlFlowTok}[1]{\textcolor[rgb]{0.13,0.29,0.53}{\textbf{#1}}}
\newcommand{\DataTypeTok}[1]{\textcolor[rgb]{0.13,0.29,0.53}{#1}}
\newcommand{\DecValTok}[1]{\textcolor[rgb]{0.00,0.00,0.81}{#1}}
\newcommand{\DocumentationTok}[1]{\textcolor[rgb]{0.56,0.35,0.01}{\textbf{\textit{#1}}}}
\newcommand{\ErrorTok}[1]{\textcolor[rgb]{0.64,0.00,0.00}{\textbf{#1}}}
\newcommand{\ExtensionTok}[1]{#1}
\newcommand{\FloatTok}[1]{\textcolor[rgb]{0.00,0.00,0.81}{#1}}
\newcommand{\FunctionTok}[1]{\textcolor[rgb]{0.00,0.00,0.00}{#1}}
\newcommand{\ImportTok}[1]{#1}
\newcommand{\InformationTok}[1]{\textcolor[rgb]{0.56,0.35,0.01}{\textbf{\textit{#1}}}}
\newcommand{\KeywordTok}[1]{\textcolor[rgb]{0.13,0.29,0.53}{\textbf{#1}}}
\newcommand{\NormalTok}[1]{#1}
\newcommand{\OperatorTok}[1]{\textcolor[rgb]{0.81,0.36,0.00}{\textbf{#1}}}
\newcommand{\OtherTok}[1]{\textcolor[rgb]{0.56,0.35,0.01}{#1}}
\newcommand{\PreprocessorTok}[1]{\textcolor[rgb]{0.56,0.35,0.01}{\textit{#1}}}
\newcommand{\RegionMarkerTok}[1]{#1}
\newcommand{\SpecialCharTok}[1]{\textcolor[rgb]{0.00,0.00,0.00}{#1}}
\newcommand{\SpecialStringTok}[1]{\textcolor[rgb]{0.31,0.60,0.02}{#1}}
\newcommand{\StringTok}[1]{\textcolor[rgb]{0.31,0.60,0.02}{#1}}
\newcommand{\VariableTok}[1]{\textcolor[rgb]{0.00,0.00,0.00}{#1}}
\newcommand{\VerbatimStringTok}[1]{\textcolor[rgb]{0.31,0.60,0.02}{#1}}
\newcommand{\WarningTok}[1]{\textcolor[rgb]{0.56,0.35,0.01}{\textbf{\textit{#1}}}}
\usepackage{graphicx}
\makeatletter
\def\maxwidth{\ifdim\Gin@nat@width>\linewidth\linewidth\else\Gin@nat@width\fi}
\def\maxheight{\ifdim\Gin@nat@height>\textheight\textheight\else\Gin@nat@height\fi}
\makeatother
% Scale images if necessary, so that they will not overflow the page
% margins by default, and it is still possible to overwrite the defaults
% using explicit options in \includegraphics[width, height, ...]{}
\setkeys{Gin}{width=\maxwidth,height=\maxheight,keepaspectratio}
% Set default figure placement to htbp
\makeatletter
\def\fps@figure{htbp}
\makeatother
\setlength{\emergencystretch}{3em} % prevent overfull lines
\providecommand{\tightlist}{%
  \setlength{\itemsep}{0pt}\setlength{\parskip}{0pt}}
\setcounter{secnumdepth}{-\maxdimen} % remove section numbering
\ifLuaTeX
  \usepackage{selnolig}  % disable illegal ligatures
\fi

\begin{document}
\maketitle

\hypertarget{working-with-data-in-r-environment}{%
\subsection{Working with data in R
environment}\label{working-with-data-in-r-environment}}

\hypertarget{a-consider-chi-square-distribution}{%
\subsubsection{a: consider Chi-Square
distribution}\label{a-consider-chi-square-distribution}}

\hypertarget{i-the-density-function-of-it-is-shown-below}{%
\paragraph{i: The density function of it is shown
below:}\label{i-the-density-function-of-it-is-shown-below}}

\begin{Shaded}
\begin{Highlighting}[]
\FunctionTok{curve}\NormalTok{(}\FunctionTok{dchisq}\NormalTok{(x, }\AttributeTok{df =} \DecValTok{11}\NormalTok{), }\AttributeTok{from =} \DecValTok{0}\NormalTok{, }\AttributeTok{to =} \DecValTok{100}\NormalTok{,}
      \AttributeTok{main =} \StringTok{\textquotesingle{}Chi{-}Square Distribution (df = 11)\textquotesingle{}}\NormalTok{, }\CommentTok{\#title}
      \AttributeTok{ylab =} \StringTok{\textquotesingle{}Density\textquotesingle{}}\NormalTok{)}
\end{Highlighting}
\end{Shaded}

\includegraphics{statistics_files/figure-latex/cars-1.pdf}

\hypertarget{ii-lets-produce-n-10-100-and-1000-samples-of-size-50-from-this-distribution.-to-do-so-we-generate-n-50-samples-and-store-them-in-a-matrix-with-n-rows-and-50-columns.}{%
\paragraph{ii: Let's produce n = 10, 100 and 1000 samples of size 50
from this distribution. To do so, we generate n * 50 samples and store
them in a matrix with n rows and 50
columns.}\label{ii-lets-produce-n-10-100-and-1000-samples-of-size-50-from-this-distribution.-to-do-so-we-generate-n-50-samples-and-store-them-in-a-matrix-with-n-rows-and-50-columns.}}

\hypertarget{for-n-10}{%
\subparagraph{for n = 10:}\label{for-n-10}}

\begin{Shaded}
\begin{Highlighting}[]
\NormalTok{samples\_1 }\OtherTok{=} \FunctionTok{rchisq}\NormalTok{(}\DecValTok{500}\NormalTok{, }\AttributeTok{df =} \DecValTok{11}\NormalTok{)}
\NormalTok{matrix\_1 }\OtherTok{=}  \FunctionTok{matrix}\NormalTok{(samples\_1, }\AttributeTok{nrow =} \DecValTok{10}\NormalTok{, }\AttributeTok{ncol =} \DecValTok{50}\NormalTok{)}
\end{Highlighting}
\end{Shaded}

\hypertarget{for-n-100}{%
\subparagraph{for n = 100:}\label{for-n-100}}

\begin{Shaded}
\begin{Highlighting}[]
\NormalTok{samples\_2 }\OtherTok{=} \FunctionTok{rchisq}\NormalTok{(}\DecValTok{5000}\NormalTok{, }\AttributeTok{df =} \DecValTok{11}\NormalTok{)}
\NormalTok{matrix\_2 }\OtherTok{=}  \FunctionTok{matrix}\NormalTok{(samples\_2, }\AttributeTok{nrow =} \DecValTok{100}\NormalTok{, }\AttributeTok{ncol =} \DecValTok{50}\NormalTok{)}
\end{Highlighting}
\end{Shaded}

\hypertarget{for-n-1000}{%
\subparagraph{for n = 1000:}\label{for-n-1000}}

\begin{Shaded}
\begin{Highlighting}[]
\NormalTok{samples\_3 }\OtherTok{=} \FunctionTok{rchisq}\NormalTok{(}\DecValTok{50000}\NormalTok{, }\AttributeTok{df =} \DecValTok{11}\NormalTok{)}
\NormalTok{matrix\_3 }\OtherTok{=}  \FunctionTok{matrix}\NormalTok{(samples\_3, }\AttributeTok{nrow =} \DecValTok{1000}\NormalTok{, }\AttributeTok{ncol =} \DecValTok{50}\NormalTok{)}
\end{Highlighting}
\end{Shaded}

\hypertarget{then-calculate-the-means-of-each-of-them-and-plot-their-histograms-note-that-we-should-calculate-the-sample-means-of-each-row}{%
\subparagraph{Then calculate the means of each of them and plot their
histograms (note that we should calculate the sample means of each
row):}\label{then-calculate-the-means-of-each-of-them-and-plot-their-histograms-note-that-we-should-calculate-the-sample-means-of-each-row}}

\begin{Shaded}
\begin{Highlighting}[]
\NormalTok{means\_1 }\OtherTok{=} \FunctionTok{apply}\NormalTok{(matrix\_1, }\DecValTok{1}\NormalTok{, mean) }\CommentTok{\# use 1 to apply it row wise}
\NormalTok{means\_2 }\OtherTok{=} \FunctionTok{apply}\NormalTok{(matrix\_2, }\DecValTok{1}\NormalTok{, mean)}
\NormalTok{means\_3 }\OtherTok{=} \FunctionTok{apply}\NormalTok{(matrix\_3, }\DecValTok{1}\NormalTok{, mean)}
\end{Highlighting}
\end{Shaded}

\hypertarget{plot-the-histograms-with-hist-function}{%
\subparagraph{plot the histograms with hist
function}\label{plot-the-histograms-with-hist-function}}

\begin{Shaded}
\begin{Highlighting}[]
\FunctionTok{hist}\NormalTok{(means\_1)}
\end{Highlighting}
\end{Shaded}

\includegraphics{statistics_files/figure-latex/unnamed-chunk-5-1.pdf}

\begin{Shaded}
\begin{Highlighting}[]
\FunctionTok{hist}\NormalTok{(means\_2)}
\end{Highlighting}
\end{Shaded}

\includegraphics{statistics_files/figure-latex/unnamed-chunk-5-2.pdf}

\begin{Shaded}
\begin{Highlighting}[]
\FunctionTok{hist}\NormalTok{(means\_3)}
\end{Highlighting}
\end{Shaded}

\includegraphics{statistics_files/figure-latex/unnamed-chunk-5-3.pdf}

\hypertarget{iii-lets-calculate-the-mean-and-variance-of-these-samples-and-compare-the-histogram-of-their-means-with-the-normal-distribution}{%
\paragraph{iii: Let's calculate the mean and variance of these samples
and compare the histogram of their means with the normal
distribution:}\label{iii-lets-calculate-the-mean-and-variance-of-these-samples-and-compare-the-histogram-of-their-means-with-the-normal-distribution}}

\begin{Shaded}
\begin{Highlighting}[]
\NormalTok{mean\_1 }\OtherTok{=} \FunctionTok{mean}\NormalTok{(means\_1)}
\NormalTok{mean\_2 }\OtherTok{=} \FunctionTok{mean}\NormalTok{(means\_2)}
\NormalTok{mean\_3 }\OtherTok{=} \FunctionTok{mean}\NormalTok{(means\_3)}
\NormalTok{mean\_1}
\end{Highlighting}
\end{Shaded}

\begin{verbatim}
## [1] 11.13094
\end{verbatim}

\begin{Shaded}
\begin{Highlighting}[]
\NormalTok{mean\_2}
\end{Highlighting}
\end{Shaded}

\begin{verbatim}
## [1] 11.02692
\end{verbatim}

\begin{Shaded}
\begin{Highlighting}[]
\NormalTok{mean\_3}
\end{Highlighting}
\end{Shaded}

\begin{verbatim}
## [1] 10.96826
\end{verbatim}

\begin{Shaded}
\begin{Highlighting}[]
\NormalTok{var\_1 }\OtherTok{=} \FunctionTok{var}\NormalTok{(means\_1)}
\NormalTok{var\_2 }\OtherTok{=} \FunctionTok{var}\NormalTok{(means\_2)}
\NormalTok{var\_3 }\OtherTok{=} \FunctionTok{var}\NormalTok{(means\_3)}
\NormalTok{var\_1}
\end{Highlighting}
\end{Shaded}

\begin{verbatim}
## [1] 0.4918401
\end{verbatim}

\begin{Shaded}
\begin{Highlighting}[]
\NormalTok{var\_2}
\end{Highlighting}
\end{Shaded}

\begin{verbatim}
## [1] 0.4255638
\end{verbatim}

\begin{Shaded}
\begin{Highlighting}[]
\NormalTok{var\_3}
\end{Highlighting}
\end{Shaded}

\begin{verbatim}
## [1] 0.4058518
\end{verbatim}

\hypertarget{next-normalize-the-means}{%
\subparagraph{Next, normalize the
means:}\label{next-normalize-the-means}}

\begin{Shaded}
\begin{Highlighting}[]
\NormalTok{normalized\_mean\_1 }\OtherTok{=}\NormalTok{ (means\_1 }\SpecialCharTok{{-}}\NormalTok{ mean\_1)}\SpecialCharTok{/}\FunctionTok{sqrt}\NormalTok{(var\_1)}
\NormalTok{normalized\_mean\_2 }\OtherTok{=}\NormalTok{ (means\_2 }\SpecialCharTok{{-}}\NormalTok{ mean\_2)}\SpecialCharTok{/}\FunctionTok{sqrt}\NormalTok{(var\_2)}
\NormalTok{normalized\_mean\_3 }\OtherTok{=}\NormalTok{ (means\_3 }\SpecialCharTok{{-}}\NormalTok{ mean\_3)}\SpecialCharTok{/}\FunctionTok{sqrt}\NormalTok{(var\_3)}
\end{Highlighting}
\end{Shaded}

\hypertarget{comapre-their-histogram-with-the-normal-dirtributions-hist}{%
\subparagraph{comapre their histogram with the normal dirtribution's
hist:}\label{comapre-their-histogram-with-the-normal-dirtributions-hist}}

\begin{Shaded}
\begin{Highlighting}[]
\CommentTok{\#hist(normalized\_mean\_1)}
\CommentTok{\#lines(seq({-}2, 1.5, by=.001), dnorm(seq({-}2, 1.5, by=.001), mean\_1, sqrt(var\_1)), col="blue")}
\CommentTok{\#hist(normalized\_mean\_2)}
\CommentTok{\#lines(seq({-}3, 2.5, by=.1), dnorm(seq({-}3, 2.5, by=.1), mean\_2, sqrt(var\_2)), col="blue")}
\CommentTok{\#hist(normalized\_mean\_3)}
\CommentTok{\#lines(seq({-}4, 4, by=.1), dnorm(seq({-}4, 4, by=.1), mean\_3, sqrt(var\_3)), col="blue")}
\FunctionTok{plotNormalHistogram}\NormalTok{( normalized\_mean\_1, }\AttributeTok{prob =} \ConstantTok{FALSE}\NormalTok{, }\AttributeTok{length =} \DecValTok{1000}\NormalTok{ )}
\end{Highlighting}
\end{Shaded}

\includegraphics{statistics_files/figure-latex/unnamed-chunk-8-1.pdf}

\begin{Shaded}
\begin{Highlighting}[]
\FunctionTok{plotNormalHistogram}\NormalTok{( normalized\_mean\_2, }\AttributeTok{prob =} \ConstantTok{FALSE}\NormalTok{, }\AttributeTok{length =} \DecValTok{1000}\NormalTok{ )}
\end{Highlighting}
\end{Shaded}

\includegraphics{statistics_files/figure-latex/unnamed-chunk-8-2.pdf}

\begin{Shaded}
\begin{Highlighting}[]
\FunctionTok{plotNormalHistogram}\NormalTok{( normalized\_mean\_3, }\AttributeTok{prob =} \ConstantTok{FALSE}\NormalTok{, }\AttributeTok{length =} \DecValTok{1000}\NormalTok{ )}
\end{Highlighting}
\end{Shaded}

\includegraphics{statistics_files/figure-latex/unnamed-chunk-8-3.pdf}

\hypertarget{we-can-see-that-as-the-n-gets-larger-the-coresponding-histogram-gets-more-similar-to-the-histogram-of-normal-distribution.-this-results-is-algined-with-the-central-limit-theorem-which-states-that-if-you-have-a-population-with-mean-ux3bc-and-standard-deviation-ux3c3-and-take-sufficiently-large-random-samples-from-the-population-with-replacement-then-the-distribution-of-the-sample-means-will-be-approximately-normally-distributed.}{%
\subparagraph{we can see that as the n gets larger, the coresponding
histogram gets more similar to the histogram of normal distribution.
This results is algined with the central limit theorem which states that
if you have a population with mean μ and standard deviation σ and take
sufficiently large random samples from the population with replacement ,
then the distribution of the sample means will be approximately normally
distributed.}\label{we-can-see-that-as-the-n-gets-larger-the-coresponding-histogram-gets-more-similar-to-the-histogram-of-normal-distribution.-this-results-is-algined-with-the-central-limit-theorem-which-states-that-if-you-have-a-population-with-mean-ux3bc-and-standard-deviation-ux3c3-and-take-sufficiently-large-random-samples-from-the-population-with-replacement-then-the-distribution-of-the-sample-means-will-be-approximately-normally-distributed.}}

\hypertarget{b-consider-the-births-data-set-from-fastr2-package-contains-inforamtion-about-the-number-of-birth-in-each-day-during-a-time-interval-of-20-years-in-america}{%
\subsubsection{b: consider the Births data set from fastR2 package
(contains inforamtion about the number of birth in each day during a
time interval of 20 years in
America)}\label{b-consider-the-births-data-set-from-fastr2-package-contains-inforamtion-about-the-number-of-birth-in-each-day-during-a-time-interval-of-20-years-in-america}}

\hypertarget{i-find-the-days-with-least-and-most-number-of-average-births}{%
\paragraph{i: Find the days with least and most number of average
births:}\label{i-find-the-days-with-least-and-most-number-of-average-births}}

\hypertarget{first-lets-view-births}{%
\subparagraph{first let's view Births}\label{first-lets-view-births}}

\begin{Shaded}
\begin{Highlighting}[]
\FunctionTok{View}\NormalTok{(Births)}
\end{Highlighting}
\end{Shaded}

\hypertarget{we-can-see-that-it-has-a-column-births-and-data-in-it.-we-iterate-over-the-days-in-those-years-find-the-mean-number-of-births-and-use-which-to-indicate-the-place-minmax-occurs.}{%
\subparagraph{we can see that it has a column births and data in it. We
iterate over the days in those years, find the mean number of births and
use which to indicate the place min/max
occurs.}\label{we-can-see-that-it-has-a-column-births-and-data-in-it.-we-iterate-over-the-days-in-those-years-find-the-mean-number-of-births-and-use-which-to-indicate-the-place-minmax-occurs.}}

\begin{Shaded}
\begin{Highlighting}[]
\NormalTok{days\_avr }\OtherTok{\textless{}{-}} \FunctionTok{matrix}\NormalTok{(}\DecValTok{0}\NormalTok{, }\DecValTok{1}\NormalTok{, }\DecValTok{365}\NormalTok{)}
\ControlFlowTok{for}\NormalTok{ (i }\ControlFlowTok{in} \DecValTok{1}\SpecialCharTok{:}\DecValTok{365}\NormalTok{) \{}
\NormalTok{  days\_avr[i] }\OtherTok{=} \FunctionTok{mean}\NormalTok{(Births}\SpecialCharTok{$}\NormalTok{births[Births}\SpecialCharTok{$}\NormalTok{day\_of\_year }\SpecialCharTok{==}\NormalTok{ i])}
\NormalTok{\}}

\FunctionTok{which.min}\NormalTok{(days\_avr)}
\end{Highlighting}
\end{Shaded}

\begin{verbatim}
## [1] 360
\end{verbatim}

\begin{Shaded}
\begin{Highlighting}[]
\FunctionTok{which.max}\NormalTok{(days\_avr)}
\end{Highlighting}
\end{Shaded}

\begin{verbatim}
## [1] 260
\end{verbatim}

\hypertarget{the-mean-value-of-each-month-is-obtained-as-follows}{%
\paragraph{the mean value of each month is obtained as
follows:}\label{the-mean-value-of-each-month-is-obtained-as-follows}}

\begin{Shaded}
\begin{Highlighting}[]
\NormalTok{months\_means }\OtherTok{\textless{}{-}} \FunctionTok{matrix}\NormalTok{(}\DecValTok{0}\NormalTok{, }\DecValTok{1}\NormalTok{, }\DecValTok{12}\NormalTok{)}
\ControlFlowTok{for}\NormalTok{ (i }\ControlFlowTok{in} \DecValTok{1}\SpecialCharTok{:}\DecValTok{12}\NormalTok{) \{}
\NormalTok{  months\_means[i] }\OtherTok{=} \FunctionTok{mean}\NormalTok{(Births}\SpecialCharTok{$}\NormalTok{births[Births}\SpecialCharTok{$}\NormalTok{month }\SpecialCharTok{==}\NormalTok{ i])}
\NormalTok{\}}
\NormalTok{months\_means}
\end{Highlighting}
\end{Shaded}

\begin{verbatim}
##          [,1]     [,2]     [,3]     [,4] [,5]     [,6]     [,7]     [,8]
## [1,] 9288.976 9491.301 9464.742 9267.958 9331 9597.618 10032.73 10177.04
##          [,9]    [,10]    [,11]    [,12]
## [1,] 10343.29 9766.729 9492.122 9523.187
\end{verbatim}

\hypertarget{also-lets-sort-the-array-obove-ascending}{%
\subparagraph{also lets sort the array obove
(ascending):}\label{also-lets-sort-the-array-obove-ascending}}

\begin{Shaded}
\begin{Highlighting}[]
\FunctionTok{order}\NormalTok{(months\_means)}
\end{Highlighting}
\end{Shaded}

\begin{verbatim}
##  [1]  4  1  5  3  2 11 12  6 10  7  8  9
\end{verbatim}

\hypertarget{to-have-it-descending-consider}{%
\subparagraph{to have it descending
consider}\label{to-have-it-descending-consider}}

\begin{Shaded}
\begin{Highlighting}[]
\FunctionTok{order}\NormalTok{(}\SpecialCharTok{{-}}\NormalTok{months\_means)}
\end{Highlighting}
\end{Shaded}

\begin{verbatim}
##  [1]  9  8  7 10  6 12 11  2  3  5  1  4
\end{verbatim}

\hypertarget{also-find-the-average-value-of-each-year-this-way}{%
\paragraph{also find the average value of each year this
way:}\label{also-find-the-average-value-of-each-year-this-way}}

\begin{Shaded}
\begin{Highlighting}[]
\NormalTok{years\_avr }\OtherTok{\textless{}{-}} \FunctionTok{matrix}\NormalTok{(}\DecValTok{0}\NormalTok{, }\DecValTok{1}\NormalTok{, }\DecValTok{20}\NormalTok{)}

\CommentTok{\#colnames(years\_avr)=c(\textquotesingle{}1969\textquotesingle{}, \textquotesingle{}1970\textquotesingle{}, \textquotesingle{}1971\textquotesingle{}, \textquotesingle{}1972\textquotesingle{}, \textquotesingle{}1973\textquotesingle{}, \textquotesingle{}1974\textquotesingle{}, \textquotesingle{}1975\textquotesingle{}, \textquotesingle{}1976\textquotesingle{}, \textquotesingle{}1977\textquotesingle{}, \textquotesingle{}1978\textquotesingle{}, \textquotesingle{}1979\textquotesingle{}, \textquotesingle{}1980\textquotesingle{}, \textquotesingle{}1981\textquotesingle{}, \textquotesingle{}1982\textquotesingle{}, \textquotesingle{}1983\textquotesingle{}, \textquotesingle{}1984\textquotesingle{}, \textquotesingle{}1985\textquotesingle{}, \textquotesingle{}1986\textquotesingle{}, \textquotesingle{}1987\textquotesingle{}, \textquotesingle{}1988\textquotesingle{})}
\ControlFlowTok{for}\NormalTok{ (i }\ControlFlowTok{in} \DecValTok{1}\SpecialCharTok{:}\DecValTok{20}\NormalTok{) \{}
\NormalTok{  years\_avr[i] }\OtherTok{=} \FunctionTok{mean}\NormalTok{(Births}\SpecialCharTok{$}\NormalTok{births[Births}\SpecialCharTok{$}\NormalTok{year }\SpecialCharTok{==}\NormalTok{ i }\SpecialCharTok{+} \DecValTok{1968}\NormalTok{])}
\NormalTok{\}}
\NormalTok{years\_avr}
\end{Highlighting}
\end{Shaded}

\begin{verbatim}
##          [,1]    [,2]     [,3]     [,4]     [,5]    [,6]     [,7]     [,8]
## [1,] 9859.118 10231.8 9761.847 8923.123 8618.274 8685.77 8639.359 8678.322
##          [,9]    [,10]    [,11]    [,12]    [,13]    [,14]    [,15]    [,16]
## [1,] 9128.899 9145.745 9588.321 9885.104 9960.219 10096.92 9980.085 10036.97
##         [,17]    [,18]    [,19]   [,20]
## [1,] 10315.22 10303.24 10447.15 10693.4
\end{verbatim}

\begin{Shaded}
\begin{Highlighting}[]
\CommentTok{\# plot the result}
\FunctionTok{ggplot}\NormalTok{(}\AttributeTok{data=}\FunctionTok{data.frame}\NormalTok{(}\AttributeTok{u=}\DecValTok{1969}\SpecialCharTok{:}\DecValTok{1988}\NormalTok{,}\AttributeTok{v=}\FunctionTok{as.vector}\NormalTok{(years\_avr))}
\NormalTok{       ,}\FunctionTok{aes}\NormalTok{(}\AttributeTok{x=}\NormalTok{u,}\AttributeTok{y=}\NormalTok{v))}\SpecialCharTok{+}\FunctionTok{geom\_bar}\NormalTok{(}\AttributeTok{stat =} \StringTok{"identity"}\NormalTok{)}
\end{Highlighting}
\end{Shaded}

\includegraphics{statistics_files/figure-latex/unnamed-chunk-14-1.pdf}

\hypertarget{c-consider-the-storms-data-set-from-dplyr-package.}{%
\subsubsection{c: consider the storms data set from dplyr
package.}\label{c-consider-the-storms-data-set-from-dplyr-package.}}

\begin{Shaded}
\begin{Highlighting}[]
\FunctionTok{View}\NormalTok{(storms)}
\end{Highlighting}
\end{Shaded}

\hypertarget{sort-them-based-on-the-time-they-occured-and-save-them-in-a-file-q1_3.txt}{%
\paragraph{sort them based on the time they occured and save them in a
file
(q1\_3.txt)}\label{sort-them-based-on-the-time-they-occured-and-save-them-in-a-file-q1_3.txt}}

\begin{Shaded}
\begin{Highlighting}[]
\NormalTok{sorted }\OtherTok{=} 
\NormalTok{  storms[}\FunctionTok{with}\NormalTok{(storms, }\FunctionTok{order}\NormalTok{(storms}\SpecialCharTok{$}\NormalTok{year, storms}\SpecialCharTok{$}\NormalTok{month, storms}\SpecialCharTok{$}\NormalTok{day , storms}\SpecialCharTok{$}\NormalTok{hour)),]}
\NormalTok{sorted}
\end{Highlighting}
\end{Shaded}

\begin{verbatim}
## # A tibble: 10,010 x 13
##    name   year month   day  hour   lat  long status      category  wind pressure
##    <chr> <dbl> <dbl> <int> <dbl> <dbl> <dbl> <chr>       <ord>    <int>    <int>
##  1 Amy    1975     6    27     0  27.5 -79   tropical d~ -1          25     1013
##  2 Amy    1975     6    27     6  28.5 -79   tropical d~ -1          25     1013
##  3 Amy    1975     6    27    12  29.5 -79   tropical d~ -1          25     1013
##  4 Amy    1975     6    27    18  30.5 -79   tropical d~ -1          25     1013
##  5 Amy    1975     6    28     0  31.5 -78.8 tropical d~ -1          25     1012
##  6 Amy    1975     6    28     6  32.4 -78.7 tropical d~ -1          25     1012
##  7 Amy    1975     6    28    12  33.3 -78   tropical d~ -1          25     1011
##  8 Amy    1975     6    28    18  34   -77   tropical d~ -1          30     1006
##  9 Amy    1975     6    29     0  34.4 -75.8 tropical s~ 0           35     1004
## 10 Amy    1975     6    29     6  34   -74.8 tropical s~ 0           40     1002
## # ... with 10,000 more rows, and 2 more variables: ts_diameter <dbl>,
## #   hu_diameter <dbl>
\end{verbatim}

\begin{Shaded}
\begin{Highlighting}[]
\FunctionTok{write.table}\NormalTok{(sorted, }\AttributeTok{file=}\StringTok{"q1\_3.txt"}\NormalTok{, }\AttributeTok{append =} \ConstantTok{FALSE}\NormalTok{, }\AttributeTok{sep =} \StringTok{" "}\NormalTok{, }\AttributeTok{dec =} \StringTok{"."}\NormalTok{,}
            \AttributeTok{row.names =} \ConstantTok{TRUE}\NormalTok{, }\AttributeTok{col.names =} \ConstantTok{TRUE}\NormalTok{)}
\end{Highlighting}
\end{Shaded}

\hypertarget{plot-the-coordinates-lat-long-of-the-place-storms-occurred}{%
\paragraph{plot the coordinates (lat, long) of the place storms
occurred}\label{plot-the-coordinates-lat-long-of-the-place-storms-occurred}}

\begin{Shaded}
\begin{Highlighting}[]
\NormalTok{p}\OtherTok{=}\FunctionTok{ggplot}\NormalTok{(}\AttributeTok{data=}\NormalTok{storms)}\SpecialCharTok{+}\FunctionTok{geom\_point}\NormalTok{(}\FunctionTok{aes}\NormalTok{(long,lat))}
\NormalTok{p}\SpecialCharTok{+}\FunctionTok{theme\_dark}\NormalTok{()}
\end{Highlighting}
\end{Shaded}

\includegraphics{statistics_files/figure-latex/unnamed-chunk-17-1.pdf}

\begin{Shaded}
\begin{Highlighting}[]
\CommentTok{\#map \textless{}{-} get\_stamenmap(bbox = c(left= {-}11 , bottom = {-}11 , right = 11 , top = {-}11 ), maptype = "terrain" , zoom = 5)}
\end{Highlighting}
\end{Shaded}

\hypertarget{also-color-them-based-on-the-status}{%
\paragraph{also color them based on the
status}\label{also-color-them-based-on-the-status}}

\begin{Shaded}
\begin{Highlighting}[]
\FunctionTok{ggplot}\NormalTok{(}\AttributeTok{data =}\NormalTok{ storms)}\SpecialCharTok{+}\FunctionTok{geom\_point}\NormalTok{(}\AttributeTok{mapping =} \FunctionTok{aes}\NormalTok{(}\AttributeTok{x=}\NormalTok{lat,}\AttributeTok{y=}\NormalTok{long,}\AttributeTok{color=}\NormalTok{status))}
\end{Highlighting}
\end{Shaded}

\includegraphics{statistics_files/figure-latex/unnamed-chunk-18-1.pdf}

\hypertarget{heart-attack-prevention}{%
\subsection{Heart Attack Prevention}\label{heart-attack-prevention}}

\hypertarget{a-loading-the-data-set-and-getting-it-ready}{%
\subsubsection{a: Loading the data set and getting it
ready}\label{a-loading-the-data-set-and-getting-it-ready}}

\hypertarget{load-the-data-set-from-had.txt}{%
\paragraph{load the data set from
``had.txt''}\label{load-the-data-set-from-had.txt}}

\begin{Shaded}
\begin{Highlighting}[]
\NormalTok{heartAttackData }\OtherTok{\textless{}{-}} \FunctionTok{read.delim}\NormalTok{(}\StringTok{"had.txt"}\NormalTok{, }\AttributeTok{header=}\NormalTok{F)}
\end{Highlighting}
\end{Shaded}

\hypertarget{i-separate-the-lastresult-column-and-store-it-in-a-variable-called-label.-store-the-first-6-columns-as-features.}{%
\paragraph{i: Separate the last(result) column and store it in a
variable called label. Store the first 6 columns as
features.}\label{i-separate-the-lastresult-column-and-store-it-in-a-variable-called-label.-store-the-first-6-columns-as-features.}}

\begin{Shaded}
\begin{Highlighting}[]
\NormalTok{label }\OtherTok{=}\NormalTok{ heartAttackData[,}\FunctionTok{ncol}\NormalTok{(heartAttackData)]}
\NormalTok{features }\OtherTok{=}\NormalTok{ heartAttackData[,}\DecValTok{1}\SpecialCharTok{:}\DecValTok{6}\NormalTok{]}
\end{Highlighting}
\end{Shaded}

\hypertarget{ii-separate-the-data-into-test20-and-train80-sets.}{%
\paragraph{ii: separate the data into test(20\%) and train(80\%)
sets.}\label{ii-separate-the-data-into-test20-and-train80-sets.}}

\begin{Shaded}
\begin{Highlighting}[]
\NormalTok{n }\OtherTok{=} \FunctionTok{length}\NormalTok{(label)}
\NormalTok{test\_ind }\OtherTok{\textless{}{-}} \FunctionTok{sample}\NormalTok{(}\DecValTok{1}\SpecialCharTok{:}\NormalTok{n, }\FunctionTok{as.integer}\NormalTok{(n }\SpecialCharTok{*} \FloatTok{0.2}\NormalTok{))}
\NormalTok{test\_label }\OtherTok{=}\NormalTok{ label[test\_ind]}
\NormalTok{test\_feature }\OtherTok{=}\NormalTok{ features[test\_ind,]}
\NormalTok{train\_label }\OtherTok{=}\NormalTok{ label[}\SpecialCharTok{{-}}\NormalTok{test\_ind]}
\NormalTok{train\_feature }\OtherTok{=}\NormalTok{ features[}\SpecialCharTok{{-}}\NormalTok{test\_ind,]}
\end{Highlighting}
\end{Shaded}

\hypertarget{save-test-and-train-sets-in-separate-files}{%
\paragraph{save test and train sets in separate
files}\label{save-test-and-train-sets-in-separate-files}}

\begin{Shaded}
\begin{Highlighting}[]
\FunctionTok{write.table}\NormalTok{(test\_label, }\AttributeTok{file=}\StringTok{"test\_label.txt"}\NormalTok{)}
\FunctionTok{write.table}\NormalTok{(test\_feature, }\AttributeTok{file=}\StringTok{"test\_feature.txt"}\NormalTok{)}
\FunctionTok{write.table}\NormalTok{(train\_label, }\AttributeTok{file=}\StringTok{"train\_label.txt"}\NormalTok{)}
\FunctionTok{write.table}\NormalTok{(train\_feature, }\AttributeTok{file=}\StringTok{"train\_feature.txt"}\NormalTok{)}
\end{Highlighting}
\end{Shaded}

\hypertarget{b-regresssion}{%
\subsubsection{b: regresssion}\label{b-regresssion}}

\hypertarget{i-use-the-train-data-set-and-apply-regression-on-them-to-obtain-the-corresponding-coefficients.}{%
\paragraph{i: use the train data set and apply regression on them to
obtain the corresponding
coefficients.}\label{i-use-the-train-data-set-and-apply-regression-on-them-to-obtain-the-corresponding-coefficients.}}

\begin{Shaded}
\begin{Highlighting}[]
\NormalTok{model }\OtherTok{\textless{}{-}} \FunctionTok{lm}\NormalTok{(}\FunctionTok{as.vector}\NormalTok{(train\_label) }\SpecialCharTok{\textasciitilde{}}  \FunctionTok{as.vector}\NormalTok{(train\_feature[,}\DecValTok{1}\NormalTok{]) }\SpecialCharTok{+} \FunctionTok{as.vector}\NormalTok{(train\_feature[,}\DecValTok{2}\NormalTok{]) }\SpecialCharTok{+} \FunctionTok{as.vector}\NormalTok{(train\_feature[,}\DecValTok{3}\NormalTok{]) }\SpecialCharTok{+} \FunctionTok{as.vector}\NormalTok{(train\_feature[,}\DecValTok{4}\NormalTok{]) }\SpecialCharTok{+} \FunctionTok{as.vector}\NormalTok{(train\_feature[,}\DecValTok{5}\NormalTok{]) }\SpecialCharTok{+} \FunctionTok{as.vector}\NormalTok{(train\_feature[,}\DecValTok{6}\NormalTok{]))}
\FunctionTok{summary}\NormalTok{(model)}
\end{Highlighting}
\end{Shaded}

\begin{verbatim}
## 
## Call:
## lm(formula = as.vector(train_label) ~ as.vector(train_feature[, 
##     1]) + as.vector(train_feature[, 2]) + as.vector(train_feature[, 
##     3]) + as.vector(train_feature[, 4]) + as.vector(train_feature[, 
##     5]) + as.vector(train_feature[, 6]))
## 
## Residuals:
##     Min      1Q  Median      3Q     Max 
## -2.1468 -0.5273 -0.0989  0.6321  1.8632 
## 
## Coefficients:
##                                Estimate Std. Error t value Pr(>|t|)    
## (Intercept)                    2.097169   0.522210   4.016 7.96e-05 ***
## as.vector(train_feature[, 1]) -0.016002   0.006136  -2.608  0.00969 ** 
## as.vector(train_feature[, 2]) -0.576178   0.118823  -4.849 2.25e-06 ***
## as.vector(train_feature[, 3])  0.434192   0.051099   8.497 2.22e-15 ***
## as.vector(train_feature[, 4]) -0.006576   0.003232  -2.035  0.04301 *  
## as.vector(train_feature[, 5]) -0.001210   0.001009  -1.199  0.23161    
## as.vector(train_feature[, 6]) -0.106537   0.154481  -0.690  0.49110    
## ---
## Signif. codes:  0 '***' 0.001 '**' 0.01 '*' 0.05 '.' 0.1 ' ' 1
## 
## Residual standard error: 0.825 on 236 degrees of freedom
## Multiple R-squared:  0.3364, Adjusted R-squared:  0.3195 
## F-statistic: 19.94 on 6 and 236 DF,  p-value: < 2.2e-16
\end{verbatim}

\hypertarget{call-shows-us-the-formula-that-r-used-to-fit-the-regression-model.-label-is-our-dependent-variable-and-we-are-using-the-6-features-as-independent-variables-to-predict.}{%
\subparagraph{Call: shows us the formula that R used to fit the
regression model. label is our dependent variable and we are using the 6
features as independent variables (to
predict).}\label{call-shows-us-the-formula-that-r-used-to-fit-the-regression-model.-label-is-our-dependent-variable-and-we-are-using-the-6-features-as-independent-variables-to-predict.}}

\hypertarget{residuals-the-residuals-are-the-difference-between-the-actual-values-and-the-predicted-ones.-wed-definitely-want-our-median-value-to-be-centered-around-zero-as-this-would-tell-us-our-residuals-were-somewhat-symmetrical-and-that-our-model-was-predicting-evenly-at-both-the-high-and-low-ends-of-our-dataset.-looking-at-the-output-above-it-looks-like-our-distribution-is-slightly-right-skewed-but-again-it-is-also-near-0-so-we-can-be-hopeful-that-our-model-is-predicting-quite-well.}{%
\subparagraph{Residuals: The residuals are the difference between the
actual values and the predicted ones. we'd definitely want our median
value to be centered around zero as this would tell us our residuals
were somewhat symmetrical and that our model was predicting evenly at
both the high and low ends of our dataset. Looking at the output above,
it looks like our distribution is slightly right-skewed but again it is
also near 0 so we can be hopeful that our model is predicting quite
well.}\label{residuals-the-residuals-are-the-difference-between-the-actual-values-and-the-predicted-ones.-wed-definitely-want-our-median-value-to-be-centered-around-zero-as-this-would-tell-us-our-residuals-were-somewhat-symmetrical-and-that-our-model-was-predicting-evenly-at-both-the-high-and-low-ends-of-our-dataset.-looking-at-the-output-above-it-looks-like-our-distribution-is-slightly-right-skewed-but-again-it-is-also-near-0-so-we-can-be-hopeful-that-our-model-is-predicting-quite-well.}}

\hypertarget{coefficients-these-are-estimations-of-coefficients-in-the-equation-of-form-labelc_1-f_1c_2-f_2c_3-f_3c_4-f_4c_5-f_5c_6-f_6c_7-where-c_7-is-the-intercept-and-c_is-are-the-coefficientslope-of-each-factor.-because-we-often-dont-have-enough-information-or-data-to-know-the-exact-equation-that-exists-in-the-wild-we-have-to-build-this-equation-by-generating-estimates-for-both-the-slopes-and-the-intercept.-these-estimates-are-most-often-generated-through-the-ordinary-least-squares-where-regression-model-finds-the-line-that-fits-the-points-in-such-a-way-that-it-minimizes-the-distance-between-each-point-and-the-line.}{%
\subparagraph{\texorpdfstring{Coefficients: These are estimations of
coefficients in the equation of form
\(label=c_1 * f_1+c_2 * f_2+c_3 * f_3+c_4 * f_4+c_5 * f_5+c_6 * f_6+c_7\),
where \(c_7\) is the intercept and \(c_i\)s are the coefficient(slope)
of each factor. Because we often don't have enough information or data
to know the exact equation that exists in the wild, we have to build
this equation by generating estimates for both the slopes and the
intercept. These estimates are most often generated through the ordinary
least squares (where regression model finds the line that fits the
points in such a way that it minimizes the distance between each point
and the
line).}{Coefficients: These are estimations of coefficients in the equation of form label=c\_1 * f\_1+c\_2 * f\_2+c\_3 * f\_3+c\_4 * f\_4+c\_5 * f\_5+c\_6 * f\_6+c\_7, where c\_7 is the intercept and c\_is are the coefficient(slope) of each factor. Because we often don't have enough information or data to know the exact equation that exists in the wild, we have to build this equation by generating estimates for both the slopes and the intercept. These estimates are most often generated through the ordinary least squares (where regression model finds the line that fits the points in such a way that it minimizes the distance between each point and the line).}}\label{coefficients-these-are-estimations-of-coefficients-in-the-equation-of-form-labelc_1-f_1c_2-f_2c_3-f_3c_4-f_4c_5-f_5c_6-f_6c_7-where-c_7-is-the-intercept-and-c_is-are-the-coefficientslope-of-each-factor.-because-we-often-dont-have-enough-information-or-data-to-know-the-exact-equation-that-exists-in-the-wild-we-have-to-build-this-equation-by-generating-estimates-for-both-the-slopes-and-the-intercept.-these-estimates-are-most-often-generated-through-the-ordinary-least-squares-where-regression-model-finds-the-line-that-fits-the-points-in-such-a-way-that-it-minimizes-the-distance-between-each-point-and-the-line.}}

estimations of each one of \(c_i\)s described above are as follows:

\(c_1=-0.0214786\), \(c_2=-0.6194578\), \(c_3=0.4163637\),
\(c_4=-0.0087373\), \(c_5=-0.0003125\), \(c_6=0.0893647\),
\(c_7=2.4708939\)

To ilustrate this in our example, I constructed a data frame.

\begin{Shaded}
\begin{Highlighting}[]
\CommentTok{\# label=as.vector(train\_label)}
\CommentTok{\# f1=as.vector(train\_feature[, 1])}
\CommentTok{\# f2=as.vector(train\_feature[, 2])}
\CommentTok{\# f3=as.vector(train\_feature[, 3])}
\CommentTok{\# f4=as.vector(train\_feature[, 4])}
\CommentTok{\# f5=as.vector(train\_feature[, 5])}
\CommentTok{\# f6=as.vector(train\_feature[, 6])}
\CommentTok{\# class\_df = data.frame(label, f1, f2, f3, f4, f5, f6, stringsAsFactors = F) }
\CommentTok{\# }
\CommentTok{\# ggplt \textless{}{-} ggplot(class\_df,aes(x=,y=age,shape=Tree))+}
\CommentTok{\#          geom\_point()+}
\CommentTok{\#          theme\_classic()}
\CommentTok{\#   }
\CommentTok{\# ggplt}
\CommentTok{\#   }
\CommentTok{\# \# Plotting multiple Regression Lines}
\CommentTok{\# ggplt+geom\_smooth(method=lm,se=FALSE,fullrange=TRUE,}
\CommentTok{\#                   aes(color=Tree))}
\end{Highlighting}
\end{Shaded}

Coefficients --- Std. Error: The standard error of the coefficient is an
estimate of the standard deviation of the coefficient. In effect, it is
telling us how much uncertainty there is with our coefficient. The
standard error is often used to create confidence intervals.

Coefficients --- t value: The t-statistic is simply the coefficient
divided by the standard error. In general, we want our coefficients to
have large t-statistics.

Coefficients --- Pr(\textgreater\textbar t\textbar) and Signif. codes
The p-value is calculated using the t-statistic from the T distribution.
The p-value, in association with the t-statistic, help us to understand
how significant our coefficient is to the model. In practice, any
p-value below 0.05 is usually deemed as significant. The coefficient
codes give us a quick way to visually see which coefficients are
significant to the model.

\hypertarget{residual-standard-error-the-residual-standard-error-is-a-measure-of-how-well-the-model-fits-the-data.the-residual-standard-error-tells-us-the-average-amount-that-the-actual-values-of-y-differ-from-the-predictions.-in-general-we-want-the-smallest-residual-standard-error-possible.}{%
\subparagraph{Residual standard error: The residual standard error is a
measure of how well the model fits the data.The residual standard error
tells us the average amount that the actual values of Y differ from the
predictions. In general, we want the smallest residual standard error
possible.}\label{residual-standard-error-the-residual-standard-error-is-a-measure-of-how-well-the-model-fits-the-data.the-residual-standard-error-tells-us-the-average-amount-that-the-actual-values-of-y-differ-from-the-predictions.-in-general-we-want-the-smallest-residual-standard-error-possible.}}

\hypertarget{multiple-r-squared-and-adjusted-r-squared-the-multiple-r-squared-value-is-most-often-used-for-simple-linear-regression-one-predictor.-it-tells-us-what-percentage-of-the-variation-within-our-dependent-variable-that-the-independent-variable-is-explaining.-its-another-method-to-determine-how-well-our-model-is-fitting-the-data.-which-in-our-example-is-a-bit-high-and-shows-that-our-model-isnt-fitting-the-data-very-well.-the-adjusted-r-squared-value-is-used-when-running-multiple-linear-regression-and-can-conceptually-be-thought-of-in-the-same-way-we-described-multiple-r-squared.-the-adjusted-r-squared-value-shows-what-percentage-of-the-variation-within-our-dependent-variable-that-all-predictors-are-explaining.-which-is-also-32-in-this-example-and-considered-to-be-almost-high.}{%
\subparagraph{Multiple R-squared and Adjusted R-squared: The Multiple
R-squared value is most often used for simple linear regression (one
predictor). It tells us what percentage of the variation within our
dependent variable that the independent variable is explaining. it's
another method to determine how well our model is fitting the data.
Which in our example is a bit high and shows that our model isn't
fitting the data very well. The Adjusted R-squared value is used when
running multiple linear regression and can conceptually be thought of in
the same way we described Multiple R-squared. The Adjusted R-squared
value shows what percentage of the variation within our dependent
variable that all predictors are explaining. Which is also 32\% in this
example and considered to be almost
high.}\label{multiple-r-squared-and-adjusted-r-squared-the-multiple-r-squared-value-is-most-often-used-for-simple-linear-regression-one-predictor.-it-tells-us-what-percentage-of-the-variation-within-our-dependent-variable-that-the-independent-variable-is-explaining.-its-another-method-to-determine-how-well-our-model-is-fitting-the-data.-which-in-our-example-is-a-bit-high-and-shows-that-our-model-isnt-fitting-the-data-very-well.-the-adjusted-r-squared-value-is-used-when-running-multiple-linear-regression-and-can-conceptually-be-thought-of-in-the-same-way-we-described-multiple-r-squared.-the-adjusted-r-squared-value-shows-what-percentage-of-the-variation-within-our-dependent-variable-that-all-predictors-are-explaining.-which-is-also-32-in-this-example-and-considered-to-be-almost-high.}}

\hypertarget{the-f-statistic-and-overall-p-value-help-us-determine-the-result-of-this-test.-if-you-have-a-lot-of-independent-variables-its-common-for-an-f-statistic-to-be-close-to-one-and-to-still-produce-a-p-value-where-we-would-reject-the-null-hypothesis.-however-for-smaller-models-a-larger-f-statistic-generally-indicates-that-the-null-hypothesis-should-be-rejected.}{%
\subparagraph{The F-statistic and overall p-value help us determine the
result of this test. If you have a lot of independent variables, it's
common for an F-statistic to be close to one and to still produce a
p-value where we would reject the null hypothesis. However, for smaller
models, a larger F-statistic generally indicates that the null
hypothesis should be
rejected.}\label{the-f-statistic-and-overall-p-value-help-us-determine-the-result-of-this-test.-if-you-have-a-lot-of-independent-variables-its-common-for-an-f-statistic-to-be-close-to-one-and-to-still-produce-a-p-value-where-we-would-reject-the-null-hypothesis.-however-for-smaller-models-a-larger-f-statistic-generally-indicates-that-the-null-hypothesis-should-be-rejected.}}

\hypertarget{ii-indentify-the-importance-of-coeffs}{%
\paragraph{ii: Indentify the importance of
coeffs:}\label{ii-indentify-the-importance-of-coeffs}}

\hypertarget{althogh-the-coeffs-show-us-the-amount-of-importance-we-can-not-just-compare-their-amounts-with-each-other-since-they-dont-have-the-same-scales-they-are-not-scaled-the-same.-we-should-first-normalize-them-for-example-between-0-and-1-and-then-compare-them-with-each-other.}{%
\subparagraph{Althogh the coeffs show us the amount of importance, we
can not just compare their amounts with each other since they don't have
the same scales they are not scaled the same. We should first normalize
them (for example between 0 and 1) and then compare them with each
other.}\label{althogh-the-coeffs-show-us-the-amount-of-importance-we-can-not-just-compare-their-amounts-with-each-other-since-they-dont-have-the-same-scales-they-are-not-scaled-the-same.-we-should-first-normalize-them-for-example-between-0-and-1-and-then-compare-them-with-each-other.}}

\hypertarget{regression-analysis-is-a-form-of-inferential-statistics.-the-p-values-help-determine-whether-the-relationships-that-you-observe-in-your-sample-also-exist-in-the-larger-population.-the-p-value-for-each-independent-variable-tests-the-null-hypothesis-that-the-variable-has-no-correlation-with-the-dependent-variable.-if-there-is-no-correlation-there-is-no-association-between-the-changes-in-the-independent-variable-and-the-shifts-in-the-dependent-variable.}{%
\subparagraph{Regression analysis is a form of inferential statistics.
The p-values help determine whether the relationships that you observe
in your sample also exist in the larger population. The p-value for each
independent variable tests the null hypothesis that the variable has no
correlation with the dependent variable. If there is no correlation,
there is no association between the changes in the independent variable
and the shifts in the dependent
variable.}\label{regression-analysis-is-a-form-of-inferential-statistics.-the-p-values-help-determine-whether-the-relationships-that-you-observe-in-your-sample-also-exist-in-the-larger-population.-the-p-value-for-each-independent-variable-tests-the-null-hypothesis-that-the-variable-has-no-correlation-with-the-dependent-variable.-if-there-is-no-correlation-there-is-no-association-between-the-changes-in-the-independent-variable-and-the-shifts-in-the-dependent-variable.}}

\hypertarget{if-the-p-value-for-a-variable-is-less-than-your-significance-level-your-sample-data-provide-enough-evidence-to-reject-the-null-hypothesis-for-the-entire-population.-on-the-other-hand-a-p-value-that-is-greater-than-the-significance-level-indicates-that-there-is-insufficient-evidence-in-your-sample-to-conclude-that-a-non-zero-correlation-exists.}{%
\subparagraph{If the p-value for a variable is less than your
significance level, your sample data provide enough evidence to reject
the null hypothesis for the entire population. On the other hand, a
p-value that is greater than the significance level indicates that there
is insufficient evidence in your sample to conclude that a non-zero
correlation
exists.}\label{if-the-p-value-for-a-variable-is-less-than-your-significance-level-your-sample-data-provide-enough-evidence-to-reject-the-null-hypothesis-for-the-entire-population.-on-the-other-hand-a-p-value-that-is-greater-than-the-significance-level-indicates-that-there-is-insufficient-evidence-in-your-sample-to-conclude-that-a-non-zero-correlation-exists.}}

\hypertarget{recall-the-p-value-in-the-coefficients-part-for-our-example}{%
\subparagraph{Recall the p-value in the coefficients part for our
example:}\label{recall-the-p-value-in-the-coefficients-part-for-our-example}}

\begin{Shaded}
\begin{Highlighting}[]
\FunctionTok{summary}\NormalTok{(model)}
\end{Highlighting}
\end{Shaded}

\begin{verbatim}
## 
## Call:
## lm(formula = as.vector(train_label) ~ as.vector(train_feature[, 
##     1]) + as.vector(train_feature[, 2]) + as.vector(train_feature[, 
##     3]) + as.vector(train_feature[, 4]) + as.vector(train_feature[, 
##     5]) + as.vector(train_feature[, 6]))
## 
## Residuals:
##     Min      1Q  Median      3Q     Max 
## -2.1468 -0.5273 -0.0989  0.6321  1.8632 
## 
## Coefficients:
##                                Estimate Std. Error t value Pr(>|t|)    
## (Intercept)                    2.097169   0.522210   4.016 7.96e-05 ***
## as.vector(train_feature[, 1]) -0.016002   0.006136  -2.608  0.00969 ** 
## as.vector(train_feature[, 2]) -0.576178   0.118823  -4.849 2.25e-06 ***
## as.vector(train_feature[, 3])  0.434192   0.051099   8.497 2.22e-15 ***
## as.vector(train_feature[, 4]) -0.006576   0.003232  -2.035  0.04301 *  
## as.vector(train_feature[, 5]) -0.001210   0.001009  -1.199  0.23161    
## as.vector(train_feature[, 6]) -0.106537   0.154481  -0.690  0.49110    
## ---
## Signif. codes:  0 '***' 0.001 '**' 0.01 '*' 0.05 '.' 0.1 ' ' 1
## 
## Residual standard error: 0.825 on 236 degrees of freedom
## Multiple R-squared:  0.3364, Adjusted R-squared:  0.3195 
## F-statistic: 19.94 on 6 and 236 DF,  p-value: < 2.2e-16
\end{verbatim}

\hypertarget{we-can-see-that-f5-and-f4-are-fairly-large-compared-to-a-0.05-significance-level-and-maybe-we-should-reconsider-using-them.}{%
\subparagraph{We can see that f5 and f4 are fairly large compared to a
0.05 significance level and maybe we should reconsider using
them.}\label{we-can-see-that-f5-and-f4-are-fairly-large-compared-to-a-0.05-significance-level-and-maybe-we-should-reconsider-using-them.}}

\hypertarget{b-evaluate-the-coeffs-on-test-set}{%
\subsubsection{b: evaluate the coeffs on test
set}\label{b-evaluate-the-coeffs-on-test-set}}

\hypertarget{apply-the-coeffs-on-test-c_1-0.0214786-c_2-0.6194578-c_30.4163637-c_4-0.0087373-c_5-0.0003125-c_60.0893647-c_72.4708939}{%
\paragraph{\texorpdfstring{apply the coeffs on test (\(c_1=-0.0214786\),
\(c_2=-0.6194578\), \(c_3=0.4163637\), \(c_4=-0.0087373\),
\(c_5=-0.0003125\), \(c_6=0.0893647\),
\(c_7=2.4708939\)):}{apply the coeffs on test (c\_1=-0.0214786, c\_2=-0.6194578, c\_3=0.4163637, c\_4=-0.0087373, c\_5=-0.0003125, c\_6=0.0893647, c\_7=2.4708939):}}\label{apply-the-coeffs-on-test-c_1-0.0214786-c_2-0.6194578-c_30.4163637-c_4-0.0087373-c_5-0.0003125-c_60.0893647-c_72.4708939}}

\begin{Shaded}
\begin{Highlighting}[]
\NormalTok{test\_result }\OtherTok{=}\NormalTok{ test\_feature[,}\DecValTok{1}\NormalTok{]}\SpecialCharTok{*{-}}\FloatTok{0.0214786} \SpecialCharTok{+}\NormalTok{ test\_feature[,}\DecValTok{2}\NormalTok{]}\SpecialCharTok{*{-}}\FloatTok{0.6194578} \SpecialCharTok{+}
\NormalTok{              test\_feature[,}\DecValTok{3}\NormalTok{]}\SpecialCharTok{*}\FloatTok{0.4163637} \SpecialCharTok{+}\NormalTok{ test\_feature[,}\DecValTok{4}\NormalTok{]}\SpecialCharTok{*{-}}\FloatTok{0.0087373} \SpecialCharTok{+}
\NormalTok{              test\_feature[,}\DecValTok{5}\NormalTok{]}\SpecialCharTok{*{-}}\FloatTok{0.0003125} \SpecialCharTok{+}\NormalTok{ test\_feature[,}\DecValTok{6}\NormalTok{]}\SpecialCharTok{*}\FloatTok{0.0893647} \SpecialCharTok{+} \FloatTok{2.4708939}
\NormalTok{test\_result}
\end{Highlighting}
\end{Shaded}

\begin{verbatim}
##  [1] -0.3453556  0.7286630  0.6886016  0.2739051 -0.1925273  1.0273435
##  [7]  0.5794427 -0.3060210  0.5793205  1.0675492 -0.0095170  0.5734543
## [13]  0.2248595  1.1167110  0.0277236 -0.2011404  0.1208081 -0.7090025
## [19] -0.1162878  0.0916379 -0.2001619  0.3900982  0.0875127 -0.4515918
## [25]  0.2386099  1.3932496  0.2041615 -0.7062362  0.8463936 -0.1677513
## [31] -0.6246308 -0.6063178  0.0578404  1.0243257 -0.4319758 -0.7520644
## [37]  0.6921840  0.3372660 -0.6100674 -0.8153164  0.6075418 -0.5487570
## [43] -0.3159037 -0.7892927 -0.3441988 -0.0036256  0.5304436  0.7539116
## [49]  0.3133654  0.2011934 -0.0519208  0.1144130 -0.3485869  0.9559737
## [55] -0.6580437 -0.1759322  0.4482741  0.4562080  1.0508620  0.7468383
\end{verbatim}

\begin{Shaded}
\begin{Highlighting}[]
\NormalTok{apply\_limit }\OtherTok{\textless{}{-}} \ControlFlowTok{function}\NormalTok{(result, limit)\{}
\NormalTok{  result[result }\SpecialCharTok{\textgreater{}}\NormalTok{ limit] }\OtherTok{=} \DecValTok{1}
\NormalTok{  result[result }\SpecialCharTok{\textless{}=}\NormalTok{ limit] }\OtherTok{=} \SpecialCharTok{{-}}\DecValTok{1}
  \FunctionTok{return}\NormalTok{(result)}
\NormalTok{\}}
\NormalTok{test\_result }\OtherTok{=} \FunctionTok{apply\_limit}\NormalTok{(test\_result, }\AttributeTok{limit =} \DecValTok{0}\NormalTok{)}
\NormalTok{test\_result}
\end{Highlighting}
\end{Shaded}

\begin{verbatim}
##  [1] -1  1  1  1 -1  1  1 -1  1  1 -1  1  1  1  1 -1  1 -1 -1  1 -1  1  1 -1  1
## [26]  1  1 -1  1 -1 -1 -1  1  1 -1 -1  1  1 -1 -1  1 -1 -1 -1 -1 -1  1  1  1  1
## [51] -1  1 -1  1 -1 -1  1  1  1  1
\end{verbatim}

\hypertarget{compare-the-test_result-with-actual-labels-to-see-how-many-are-correct}{%
\subparagraph{compare the test\_result with actual labels to see how
many are
correct:}\label{compare-the-test_result-with-actual-labels-to-see-how-many-are-correct}}

\begin{Shaded}
\begin{Highlighting}[]
\NormalTok{check\_correctness }\OtherTok{\textless{}{-}} \ControlFlowTok{function}\NormalTok{(result)\{}
\NormalTok{  corrects }\OtherTok{=} \DecValTok{0}
  \ControlFlowTok{for}\NormalTok{ (i }\ControlFlowTok{in} \DecValTok{1}\SpecialCharTok{:}\FunctionTok{length}\NormalTok{(test\_result)) \{}
    \ControlFlowTok{if}\NormalTok{ (test\_result[i] }\SpecialCharTok{==}\NormalTok{ test\_label[i]) \{}
\NormalTok{      corrects }\OtherTok{=}\NormalTok{ corrects }\SpecialCharTok{+} \DecValTok{1}
\NormalTok{    \}}
\NormalTok{  \}}
  \FunctionTok{return}\NormalTok{(corrects}\SpecialCharTok{/}\FunctionTok{length}\NormalTok{(test\_result))}
\NormalTok{\}}

\FunctionTok{check\_correctness}\NormalTok{(test\_result)}
\end{Highlighting}
\end{Shaded}

\begin{verbatim}
## [1] 0.7666667
\end{verbatim}

\hypertarget{ii-find-the-maximum-correcness-rate-for-different-values-of-limit-in--2-2}{%
\paragraph{ii: find the maximum correcness rate for different values of
limit in {[}-2,
2{]}:}\label{ii-find-the-maximum-correcness-rate-for-different-values-of-limit-in--2-2}}

\end{document}
